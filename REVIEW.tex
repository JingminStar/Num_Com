 \documentclass[10pt]{article}
% \geometry{left=1.5cm,right=1.5cm,top=1.5cm,bottom=1.5cm}
\usepackage{geometry}
\geometry{left=3cm,right=3cm,top=3cm,bottom=3cm}
\usepackage{datetime}
\usepackage{tikz}
\usepackage{color,array,graphics}
\usepackage{enumerate}
\usepackage[utf8]{inputenc} %useful to type directly diacritic characters
\usepackage{comment}
\usepackage{import}
\usepackage{CJKutf8}
\usepackage[]{algorithm2e}
\usepackage{amsthm}
\usepackage{stmaryrd}
\theoremstyle{definition}
\newtheorem{definition}{Definition}[section]
\newtheorem{theorem}{Theorem}[section]
\newtheorem{lemma}[theorem]{Lemma}
\newtheorem{corollary}[theorem]{Corollary}
\usepackage{amssymb} %maths
\usepackage{amsmath} %maths
%\numberwithin{equation}{section}
\graphicspath{ {images/} }
\usepackage{float}

\usepackage{tcolorbox}
\usepackage{hyperref}


%\tcbuselibrary{most}




% for coding
\usepackage{listings}
\usepackage{color}
\definecolor{dkgreen}{rgb}{0,0.6,0}
\definecolor{gray}{rgb}{0.5,0.5,0.5}
\definecolor{mauve}{rgb}{0.58,0,0.82}
\newcommand{\since}{$\because$}
\newcommand{\so}{$\therefore$}

\lstset{frame=tb,
	language=Python,
	aboveskip=3mm,
	belowskip=3mm,
	showstringspaces=false,
	columns=flexible,
	basicstyle={\small\ttfamily},
	numbers=none,
	numberstyle=\tiny\color{gray},
	keywordstyle=\color{blue},
	commentstyle=\color{dkgreen},
	stringstyle=\color{mauve},
	breaklines=true,
	breakatwhitespace=true,
	tabsize=4
}


\usepackage{fancyhdr}                                
\usepackage{lastpage}                                           
\usepackage{layout}                                             
%\pagestyle{empty}                   %不设置页眉页脚            
\footskip = 10pt                                                
\pagestyle{fancy}                    % 设置页眉                 
\lhead{Jingmin Sun(sunj8)}
%\chead{PHYS 4330 Theoretical Mechanics(2018)}   
\chead{MATH 4800 Numerical Computing (2019-Spring)}         
\rhead{RIN:661849071}                   
\cfoot{page \thepage\ of \pageref{LastPage}}                        
%\rfoot{root left}%                                                     
%\lfoot{root right}                                                  
\renewcommand{\headrulewidth}{2pt}  %页眉线宽,设为0可以去页眉线
\setlength{\skip\footins}{0.5cm}    %脚注与正文的距离           
\renewcommand{\footnotesize}{}      %设置脚注字体大小           
\renewcommand{\footrulewidth}{0pt}  %脚注线的宽度  
% \lhead{page \thepage\ of \pageref{LastPage}}                    
             


\setlength\parindent{0pt}   


\begin{document}

	\begin{center} 
		%\begin{minipage}{\textwidth} 
		\centering{ {\textbf{\huge{NC Lecture Notebook}}}} \\
		%\end{minipage} 
	\end{center} 
	\tableofcontents
	\pagebreak
\section{Introduction}
 	 Examples:
 	 \begin{enumerate}
 	 \item  Compute the partial sums of the harmonic series \[\sum_{k=1}^n \frac{1}{k}\]
 	 \begin{itemize}
 	 \item
 	$S(n) = 1+\frac{1}{2}+\cdots +\frac{1}{n-1}+\frac{1}{n}$
 	\item
 	$s(n) =\frac{1}{n} +\frac{1}{n-1}+\cdots +\frac{1}{2}+1$
 	 \end{itemize}
 	 
 	 Mathematically, $S(n) = s(n)$; Computationally, they're not.
 	 
 	 The difference growth with $n$
 	 
 	\item
 	Let $f(x) = \sqrt{x}$ for $x>0$, and we know $f'(x) = \dfrac{1}{2\sqrt{x}}$ 
 	
 	Define a function \begin{align*}
 	y(k) &=\dfrac{f(16+k)-f(16)}{k}\\
 	&=\dfrac{\sqrt{16+k}-4}{k}
 	\end{align*}
 then, $\lim_{k\rightarrow 0} y (k) = f'(16) = \frac{1}{8}$	
 
 As k decrease to $k = 10^{-12}$, $y(k)$ is a good approximation for $f'(16) = \dfrac{1}{8}$, when k further decrease, y(k) starts to oscillates and the  errors are visible; after k drops below $10^{-14}$, the computed $y(k)$ is around 0. 
 \[\dfrac{\sqrt{16+k}-4}{k} = \dfrac{1}{\sqrt{16+k}+4}\]$\dfrac{\sqrt{16+k}-4}{k}$ goes to 0.
 
 \item
 Consider the function $y(x) = (x-1)^8$ and it's expanded form \[y(x) = x^8 -8x^7+28x^6-56x^5+70x^4-56x^3+28x^2-8x+1\]
 
 Again, two mathematically identical function are nor the same computationally. 
 
 The expected form even leads to negative values.
 \end{enumerate} 	
 \subsection{Computational Complexity : Polynomial Evaluation}
 
 \[p(x) = 2x^4+3x^3-3x^2+5x-1\]
 \begin{enumerate}
 \item Straight Forward:
 
 \begin{align*}
 2 \cdot \frac{1}{2} \cdot  \frac{1}{2} \cdot  \frac{1}{2} \cdot  \frac{1}{2} && 4(x)\\
 3 \cdot  \frac{1}{2} \cdot  \frac{1}{2} \cdot  \frac{1}{2} && 3(x)\\
  (-3)\cdot  \frac{1}{2} \cdot  \frac{1}{2} && 2(x)\\
  5\cdot \frac{1}{2} &&1(x)
 \end{align*}
 $$N_x = 10 ,N_+=4$$
 \item Storage
 \begin{align*}
 \left(\dfrac{1}{2}\right)^2 &=\dfrac{1}{4} &&1(x)\\
   \left(\dfrac{1}{2}\right)^3 &= \left(\dfrac{1}{2}\right)^2\cdot \dfrac{1}{2} = \dfrac{1}{8} &&1(x)\\
   \left(\dfrac{1}{2}\right)^4 &=\left(\dfrac{1}{2}\right)^3\cdot \dfrac{1}{2} = \dfrac{1}{16} &&1(x)\\
 \end{align*}
 multiply by coefficient $4(x)$\\
 $$N_x = 7, N_+=4$$
 \item Horner's method:
 \begin{align*}
 P(x)&=x(2x^3+3x^2-3x+5)-1\\
 &=x(x(2x^2+3x-3)+5)-1\\
  &=x(x(x(2x+3)-3)+5)-1\\
 \end{align*}
 $$N_x = 4 , N_+=4$$
 
\begin{align*}
N_+ =4\\
N_x &=\begin{cases}
\sum_{k=1}^d k &=\dfrac{d(1+d)}{2}\\
2d - 1\\
d
\end{cases}
\end{align*}
 \end{enumerate}
   \subsection{Floating Point Arithmetic}
   Consider -321.416: (Decimal Representation)
   \begin{align*}
   -321.416&=-(3\cdot 10^2 + 2\cdot 10 +1\cdot 0 + 4\cdot 10^{-1}+1\cdot 10^{-2}+6\cdot 10^{-3})
   \end{align*}
   
   A similar representation is used in computer: floating - point arithmetic:
   $$ - .321416 \times 10^3$$
   sign, fraction, base, exponent
   
   In general, $$\pm f \times \beta^e$$
   where $\beta =2$: binary number\\
   $\beta =10$: decimal number\\
   $\beta =16$: hexadecimal number
   $f$: fraction, digits from $0,1,\cdots \beta-1$\\
   $e$: exponent, digits from $0,1,\cdots \beta-1$
   
   Binary Number: 
   \[b_m \cdots b_2b_1b_0. a_1a_2\cdots a_n\](all integer)
   
   Each digit $b_i$, $a_j$ takes $0$ or $1$.
   
   This number in base 10, is $$b_m\cdot 2^m+b_{m-1}\cdot 2^{m-1}+\cdots + b_1\cdot 2^1+b_0\cdot 2^0 + a_1\cdot 2^{-1} +  a_2\cdot 2^{-2} + \cdots +a_n2^{-n}$$
   
   Note: \begin{align*}
   (0.1101)_2 &= (1.101)_2\cdot 2^{-1}\\
   &= (0.001101)_2\cdot 2^3
   \end{align*}
   
   To convert between binary ($\beta =2$) and decimal  ($\beta =10$)
   
   Example:
   \begin{enumerate}
   \item $x = (1.1011)_2$ convert x to a decimal number:\begin{align*}
   x&=1\cdot 2^0+1\cdot 2^{-1} + 0 \cdot 2^{-2}+1\cdot 2^{-3}+1\cdot 2^{-4}\\
   &= 1+\frac{1}{2}+0+\frac{1}{8}+\frac{1}{16}\\
   &=\frac{27}{16}
   \end{align*}
   \item
   \begin{align*}
   x &= (1.1010 \cdots 10)_2\\
   &= (1.\bar{10})_2\\
   &=1\cdot 2^0+1\cdot 2^{-1} +1\cdot 2^{-3}+1\cdot 2^{-5}+\cdots
   \end{align*}
   Recall geometric series:\[1+r+r^2+\cdots = \frac{1}{1-r}\] if $|r| < 1$
   
   \begin{align*}
   x &= 1+\frac{1}{2}+\left(\dfrac{1}{2}\right)^3+\left(\dfrac{1}{2}\right)^5+\cdots\\
   &=1+\dfrac{1}{2}\left(1+\dfrac{1}{4}+\left(\dfrac{1}{4}\right)^2+\cdots\right)\\
   &=1+\dfrac{1}{2}\cdot \dfrac{1}{1-\frac{1}{4}}\\
   &=1+\frac{2}{3}\\
   &=\frac{5}{3}
   \end{align*}
   Alternatively, \begin{align*}
   x &=(1.\bar{10})_2\\
   &=1\cdot 2^0+(0.\bar{10})_2\\
   &=1+(10.\bar{10})_2\cdot  2^{-2}
   \end{align*}
   
   \begin{align*}
    y &=(0.\bar{10})_2\\
    &=(10.\bar{10})_2 \cdot 2^{-2}\\
    &=\{(10)_2\cdot 2^{-2}+(0.\bar{10})_2\cdot  2^{-2}\}\\
    y&=(2+y)\cdot 2^{-2}\\
    4y&=2+y\\
    y&=\frac{2}{3}\\
    x&=1+y\\
    &=\frac{5}{3}
   \end{align*}
   \item Convert $14.8125$ to a binary number:
   
   We are looking for \[14.8125 = (b_mb_{m-1}\cdots b_1b_0.a_1a_2\cdots a_n)_2\]
   
   Fractional part: \begin{align*}
   0.8125&= (.a_1a_2\cdots a_n)_2\\
   &=a_1\cdot 2^{-1}+a_2\cdot 2^{-2}+\cdots +a_n\cdot 2^{-n}
   \end{align*}
   \begin{itemize}
   \item $*2$
   
   \begin{align*}
   1.6250&= a_1 + a_2\cdot 2^{-1}+ \cdots a_n \cdot 2^{-(n-1)}\\
   a_1 &=1\\
     0.6250&=  a_2\cdot 2^{-1}+ \cdots a_n \cdot 2^{-(n-1)}\\
   \end{align*}
\item $*2$
\begin{align*}
1.2500 &=  a_2+ a_3\cdot 2^{-1}\cdots a_n \cdot 2^{-(n-2)}\\
a_2&=1
\end{align*}
\item $*2$
\begin{align*}
0.25 \cdot 2 &= 0.50 && a_3=0\\
0.50 \cdot 2 &=1 &&a_4=1
\end{align*}
\[\therefore 0.8125 = (.1101)_2\]
   \end{itemize}
   Collect Integer part ordered from radix point:
   
   Integer part: \begin{align*}
   14 &= (b_m\cdots b_2b_1b_0)_2\\
   &=b_m \cdot 2^m + b_{m-1} \cdot 2^{m-1} + \cdots b_1\cdot 2^1 + b_0
   \end{align*}
Divided by 2: 
\begin{align*}
\dfrac{14}{2}&= 7 R 0\\
&= (b_m\cdot 2^{m-1}+\cdots b_1)R b_0\\
\dfrac{7}{2} &= 3 R 1 &&b_1\\
\dfrac{3}{2} &= 1 R 1 &&b_2\\
\dfrac{1}{2} &= 0 R 1 &&b_3\\
\therefore 14 &=(1110)_2
\end{align*}
\[\therefore 14.8125 = (1110.1101)_2\]
\end{enumerate}  
\subsubsection{Floating point number}
$$\pm f \times \beta^e$$  

f(fraction) : the number of digits in f determines the precision.

e(exponent):the number of digits in e determines the range of representable numbers.

We follow IEEE 754 floating point standard:
\begin{enumerate}

\item Normalized form: $f = 1.b_m b_{m-1}\cdots b_1b_0$, 
$(0.0101010\cdots)$

\item
Advantage: leading 1 needs not be stored

\begin{itemize}
\item
 32-bit single precision: \\
 sign : 1 -bit \\
 exponent : 8-bits\\
 fraction: 23 -bits
 \item
 64 - bit double precision:\\
  sign : 1 -bit \\
 exponent : 11-bits\\
 fraction: 52-bits
\end{itemize}

\item
The represented number is \[(-1)^s\cdot (1+f)\cdot 2^{e-e_0}\]
\begin{itemize}
\item  e: unsigned, $e^0$: exponent bias
\item $e-e^0$: can be either positive or negative ( negative represent small number)
\end{itemize}
Let's focus on "e" or equivalently $2^{e-e_0}$
\begin{itemize}
\item Single Precision:
\begin{align*}
e &\in [e_{min}, e_{max}]\\
e_{min} &= (0\cdots 0 1)  (\text{8 bit}) =1\\ 
e_{max} &= (11\cdots 10)\\& = 1\cdot (2 + 2^2 +\cdots + 2^7)\\
&=2\cdot \left(\dfrac{1-2^7}{1-2}\right)\\
&=254\\
\implies 2^{e- e_0}& \in[2^{-126},2^{127}] &e_0 = 127\\&\approx [10^{-38},10^{38}]
\end{align*}
\item
\begin{align*}
e&\in [e_{min}, e_{max}]\\
e_{min}&=1\\
e_{max}&= 2^1+2^2+\cdots +2^{10}\\
&=2046\\
e_0 &=1023\\
2^{e-e_0}&\in [2^{-1022}, 2^{1023}]\\
&\approx[10^{-308}, 10^{308}]
\end{align*}
\end{itemize}

\subsubsection{fraction f and precision}

Using double-precision on an example: \begin{itemize}
\item How to store a number
\item How to do calculation
\end{itemize}

Consider \begin{align*}
x_1 &= \dfrac{27}{16} = (1.1011)_2\\
x_2&=\dfrac{5}{3} = (1.\bar{10})_2\\
x_3 &=\dfrac{2}{3} = (.\bar{10})_2 = (1.\bar{01})_2\cdot 2^{-1}\\
x_4 &=1 &=(1.0)_2\\
x_5 &=1\times 2^{-52}\\
x_6 &= 1\times 2^{-53}
\end{align*}

\begin{align*}
x_1 &: 1.\boxed{101100\cdots 0} (52 bits)\\
x_4 &: 1.\boxed{00\cdots 0} (52 bits)\\
x_5&: 1.\boxed{00\cdots 0} (52 bits)\times 2^{-52}\\
x_6&: 1.\boxed{00\cdots 0} (52 bits)\times 2^{-53}\\
\end{align*}
Now $x_2$, $x_3$:

\begin{align*}
x_2 &: 1.\boxed{101010\cdots 10}10\cdots\\
x_3&:1\boxed{0101\cdots 01}0101\cdots \times 2^{-1}
\end{align*}

We follow: IEEE rounding to the nearest rule:
$x\rightarrow fl(x) $
 \end{enumerate}
\end{document}



