\documentclass{article}
\usepackage{amsmath,amssymb}
\usepackage{fullpage}
\usepackage{mathrsfs}
\usepackage{setspace}
\usepackage{graphicx}
\usepackage{listings}
\usepackage{multirow}
\renewcommand{\baselinestretch}{1.1}
\pagestyle{empty}
\usepackage{color}
\definecolor{dkgreen}{rgb}{0,0.6,0}
\definecolor{gray}{rgb}{0.5,0.5,0.5}
\definecolor{mauve}{rgb}{0.58,0,0.82}

\lstset{frame=tb,
	language=Matlab,
	aboveskip=3mm,
	belowskip=3mm,
	showstringspaces=false,
	columns=flexible,
	basicstyle={\small\ttfamily},
	numbers=none,
	numberstyle=\tiny\color{gray},
	keywordstyle=\color{blue},
	commentstyle=\color{dkgreen},
	stringstyle=\color{mauve},
	breaklines=true,
	breakatwhitespace=true,
	tabsize=4
}
\begin{document}
\noindent{\bf Homework 6}

\noindent{\bf Jingmin Sun}

\noindent{\bf 661849071}


\begin{enumerate}
\item
\begin{itemize}
\item
If $b \in $ span$(A)$, then the system is solvable.
\item
When the column of $A$ is linear independent.
\item
For example, $A = \begin{bmatrix}
1&2&3\\1&3&4\\1&4&5\\1&5&6
\end{bmatrix}$, $ b = \begin{bmatrix}
3\\4\\5\\6
\end{bmatrix}$, and we can get $x =\begin{bmatrix}
1\\1\\0
\end{bmatrix}$, or $x =\begin{bmatrix}
0\\0\\1
\end{bmatrix}$
\end{itemize}
\item
Quadratic fitting: Look for $f(x) = a_0+a_1x+x_2x^2$, which means \begin{align*}
A\begin{bmatrix}
a_1\\a_2\\a_3
\end{bmatrix}&=\begin{bmatrix}
2\\2\\1\\3
\end{bmatrix}\\
\mbox{ where }A &=\begin{bmatrix}
1&1&1\\1&3&9\\1&4&16\\1&6&36
\end{bmatrix}
\end{align*}
Since we want to solve this system in the least square sense, we would like to solve the norm equation instead, which is \begin{align*}
A^TA\tilde{a}&=A^Tb\\
A^TA&=\begin{bmatrix}
1&1&1&1\\1&3&4&6\\1&9&16&36
\end{bmatrix}\begin{bmatrix}
1&1&1\\1&3&9\\1&4&16\\1&6&36
\end{bmatrix}\\
&=\begin{bmatrix}
4&14&62\\14&62&308\\62&308&1634
\end{bmatrix}\\
A^Tb&=\begin{bmatrix}
1&1&1&1\\1&3&4&6\\1&9&16&36
\end{bmatrix}\begin{bmatrix}
2\\2\\1\\3
\end{bmatrix}\\&=\begin{bmatrix}
8\\30\\144
\end{bmatrix}\\
\therefore \tilde{a}&=\begin{bmatrix}
\frac{77}{26}\\-\frac{79}{78}\\\frac{1}{6}
\end{bmatrix}
\end{align*}
Since RMSE$ = \sqrt{\dfrac{SE}{m}}$, so \begin{align*}
\sqrt{\dfrac{SE}{m}}&=\sqrt{\dfrac{||r||^2}{m}}\\
&=\sqrt{\dfrac{(Aa-b)^T(Aa-b)}{m}}\\
Aa-b&=\begin{bmatrix}
1&1&1\\1&3&9\\1&4&16\\1&6&36
\end{bmatrix}\begin{bmatrix}
\frac{77}{26}\\-\frac{79}{78}\\\frac{1}{6}
\end{bmatrix}-\begin{bmatrix}
2\\2\\1\\3
\end{bmatrix}\\
&=\begin{bmatrix}
\frac{3}{26}\\
-\frac{15}{26}\\
\frac{15}{26}\\
-\frac{3}{26}\\
\end{bmatrix}\\
\sqrt{\dfrac{SE}{m}}&=\sqrt{\dfrac{9/13}{4}}\\
&=0.416
\end{align*}
\item
We need to find $z = c_1+c_2x+c_3y$, so we can have the linear system\begin{align*}
Ac&=z\\
c=\begin{bmatrix}
c_1\\c_2\\c_3
\end{bmatrix}
&A=\begin{bmatrix}
1&0&0\\1&0&1\\1&1&0\\1&1&1\\1&1&2
\end{bmatrix} z=\begin{bmatrix}
3\\2\\3\\5\\6
\end{bmatrix}\\
\end{align*}
Since in least square sense, so \begin{align*}
A^TAc &=A^Tz\\
A^TA&=\begin{bmatrix}
1&1&1&1&1\\0&0&1&1&1\\0&1&0&1&2
\end{bmatrix}\begin{bmatrix}
1&0&0\\1&0&1\\1&1&0\\1&1&1\\1&1&2
\end{bmatrix}\\
&=\begin{bmatrix}
5&3&4\\3&3&3\\4&3&6
\end{bmatrix}\\
\end{align*}\begin{align*}
A^Tz&=\begin{bmatrix}
19\\14\\19
\end{bmatrix}\\
c &= \begin{bmatrix}
2\\\frac{5}{3}\\1
\end{bmatrix}\\
\therefore z&=2+\dfrac{5}{3}x+y
\end{align*}
\item
We need to solve $f(t) = c_1+c_2\cos 2\pi t+c_3\sin 2\pi t$, which is equivalent to solve \begin{align*}
Ac&=y\\
A &= \begin{bmatrix}
1&\cos(0)&\sin(0)\\
1&\cos(\pi)&\sin(\pi)\\
1&\cos(2\pi)&\sin(2\pi)\\
1&\cos(3\pi)&\sin(3\pi)\\
\end{bmatrix}\\
&=\begin{bmatrix}
1&1&0\\
1&-1&0\\
1&1&0\\
1&-1&0\\
\end{bmatrix}\\
c = \begin{bmatrix}
c_1\\c_2\\c_3
\end{bmatrix}&y =\begin{bmatrix}
3\\1\\3\\2
\end{bmatrix}
\end{align*}
Since we need to solve it in the least square sense, which means \begin{align*}
A^TAc&=A^Ty\\
A^TA&=\begin{bmatrix}
1&1&1&1\\1&-1&1&-1\\0&0&0&0
\end{bmatrix}\begin{bmatrix}
1&1&0\\
1&-1&0\\
1&1&0\\
1&-1&0\\
\end{bmatrix}\\
&=\begin{bmatrix}
4&0&0\\0&4&0\\0&0&0
\end{bmatrix}\\
A^Ty &=\begin{bmatrix}
1&1&1&1\\1&-1&1&-1\\0&0&0&0
\end{bmatrix} \begin{bmatrix}
3\\1\\3\\2
\end{bmatrix}\\
&=\begin{bmatrix}
9\\3\\0
\end{bmatrix}\\
\end{align*}\begin{align*}
c&=\begin{bmatrix}
\frac{9}{4}\\\frac{3}{4}\\c_3
\end{bmatrix}\\
\therefore f(t)&=\dfrac{9}{4}+\dfrac{3}{4}\cos 2 \pi t+c_3 \sin 2\pi t\\
\end{align*}
The error is \begin{align*}
r&=y-Ac\\
&=\begin{bmatrix}
3\\1\\3\\2
\end{bmatrix}-\begin{bmatrix}
1&1&0\\
1&-1&0\\
1&1&0\\
1&-1&0\\
\end{bmatrix}\begin{bmatrix}
\frac{9}{4}\\\frac{3}{4}\\c_3
\end{bmatrix}\\
&=\begin{bmatrix}
3\\1\\3\\2
\end{bmatrix}-\begin{bmatrix}
3\\\frac{3}{2}\\3\\\frac{3}{2}
\end{bmatrix}\\
&=\begin{bmatrix}
0\\-\frac{1}{2}\\0\\\frac{1}{2}
\end{bmatrix}
\end{align*}
And 2-norm error is \begin{align*}
||r||_2&=\sqrt{||r||^2}\\
&=\sqrt{\dfrac{1}{2}}\\&=\dfrac{\sqrt{2}}{2}
\end{align*}
RMSE :\begin{align*}
\sqrt{\dfrac{||r||^2}{4}}&=\sqrt{\dfrac{\frac{1}{2}}{4}}\\
&=\dfrac{\sqrt{2}}{4}
\end{align*}
\end{enumerate}

\end{document}