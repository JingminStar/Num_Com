\documentclass{article}
\usepackage{amsmath,amssymb}
\usepackage{fullpage}
\usepackage{mathrsfs}
\usepackage{setspace}
\usepackage{graphicx}
\usepackage{listings}
\renewcommand{\baselinestretch}{1.2}
\pagestyle{empty}
\usepackage{color}
\definecolor{dkgreen}{rgb}{0,0.6,0}
\definecolor{gray}{rgb}{0.5,0.5,0.5}
\definecolor{mauve}{rgb}{0.58,0,0.82}

\lstset{frame=tb,
	language=Matlab,
	aboveskip=3mm,
	belowskip=3mm,
	showstringspaces=false,
	columns=flexible,
	basicstyle={\small\ttfamily},
	numbers=none,
	numberstyle=\tiny\color{gray},
	keywordstyle=\color{blue},
	commentstyle=\color{dkgreen},
	stringstyle=\color{mauve},
	breaklines=true,
	breakatwhitespace=true,
	tabsize=4
}
\begin{document}
\noindent{\bf Homework 3}

\noindent{\bf Jingmin Sun}

\noindent{\bf 661849071}


\begin{enumerate}
\item
\begin{enumerate}
\item

\begin{align*}
g(x)&= \dfrac{2x-1}{x^2}\\
&=(2x-1)\cdot x^{-2}\\
g'(x) &=2\cdot x^{-2} -2x^{-3}\cdot(2x-1)\\
&=2\cdot x^{-2} -4x^{-2}+2x^{-3}\\
&=-2x^{-2}+2x^{-3}\\
\mbox{At the point } &x = r=1,\; |g'(1)|=|0| < 1\\
\therefore g(x)  &\mbox{ is locally convergent.} 
\end{align*}
\item
\begin{align*}
g(x)&=\cos x +\pi+1\\
g'(x)&= -\sin x \\
\mbox{At the point } &x = r=\pi,\; |g'(\pi)|=|-\sin\pi|  = 0 < 1\\
\therefore g(x)  &\mbox{ is locally convergent.} 
\end{align*}
\item
\begin{align*}
g(x)&= e^{2x}-1\\
g'(x)&= 2e^{2x}\\
\mbox{At the point } &x = r=0,\; |g'(0)|=|2|  > 1\\
\therefore g(x)  &\mbox{ is not locally convergent.}
\end{align*}
\end{enumerate}
\item To find the fixed point for $g(x) = x^2-\frac{3}{2}x +\frac{3}{2}$, which means we need to solve the equation for $x = g(x)$, which means

 \begin{align*}
x &= g(x)\\
x &=  x^2-\frac{3}{2}x +\frac{3}{2}\\
x^2-\frac{5}{2}x +\frac{3}{2}&=0\\
2x^2-5x+3&=0\\
(x-1)(2x-3) &=0\\
x_1 = 1&\; x_2 = \frac{3}{2}\\
\because g'(x) &=2x-\frac{3}{2}\\
|g'(1)|&=\left|2-\frac{3}{2}\right|\\
&=\frac{1}{2} < 1\\
\left|g'\left(\frac{3}{2}\right)\right|&=\left|3-\frac{3}{2}\right|\\
&=\frac{3}{2} >1\\
\therefore \mbox{At the point } &x = 1,\; g(x) \mbox{ is locally convergent.}\\
 \mbox{At the point } &x = \frac{3}{2},\; g(x) \mbox{ is not locally convergent.}
\end{align*}
\item
\[2x^3 -x +e^x =0\]
\begin{itemize}
\item $x = 2x^3+e^x$
\item \begin{align*}
2x^3 &= -e^x +x\\
x^3&=\dfrac{x-e^x}{2}\\
x &=\sqrt[3]{\dfrac{x-e^x}{2}}
\end{align*}
\item \begin{align*}
e^x &=x-2x^3\\
x&= \ln(x - 2x^3)
\end{align*}
\end{itemize}
\item
\begin{enumerate}
\item
Firstly, check the convergence or not:
\begin{align*}
g(x) &= \frac{1}{2}x+\frac{1}{x}\\
x &= \frac{1}{2}x+\frac{1}{x}\\
\frac{1}{2}x &=\frac{1}{x}\\
x^2 &=2\\
\end{align*}
And $x = \sqrt{2}$ is a solution follows, and we can check the convergence rate:
\begin{align*}
g'(x) &= \frac{1}{2}-\frac{1}{x^2}\\
g'(\sqrt{2}) &= \frac{1}{2}- \frac{1}{2} =0
\end{align*}
\item
Firstly, check the convergence or not:
\begin{align*}
g(x) &= \frac{2}{3}x+\frac{2}{3x}\\
x &= \frac{2}{3}x+\frac{2}{3x}\\
\frac{1}{3}x &=\frac{2}{3x}\\
x^2 &=2\\
\end{align*}
And $x = \sqrt{2}$ is a solution follows, and we can check the convergence rate:
\begin{align*}
g'(x) &= \frac{2}{3}-\frac{2}{3x^2}\\
g'(\sqrt{2}) &= \frac{2}{3}- \frac{2}{6} =\frac{1}{3}
\end{align*}
\item
Firstly, check the convergence or not:
\begin{align*}
g(x) &= \frac{3}{4}x+\frac{1}{2x}\\
x &= \frac{3}{4}x+\frac{1}{2x}\\
\frac{1}{4}x &=\frac{1}{2x}\\
x^2 &=2\\
\end{align*}
And $x = \sqrt{2}$ is a solution follows, and we can check the convergence rate:
\begin{align*}
g'(x) &= \frac{3}{4}-\frac{1}{2x^2}\\
g'(\sqrt{2}) &= \frac{3}{4}- \frac{1}{4} =\frac{1}{2}
\end{align*}
So the rank is $A > B > C$
\end{enumerate}
\item
\begin{enumerate}
\item
The original system can be expressed as \begin{align*}
\begin{bmatrix}
1&3\\5&4
\end{bmatrix}\begin{bmatrix}
u\\v
\end{bmatrix}&= \begin{bmatrix}
-1\\6
\end{bmatrix}
\end{align*}
And we can change it to diagonally dominant by \begin{align*}
\begin{bmatrix}
5&4\\ 1&3
\end{bmatrix}\begin{bmatrix}
u\\v
\end{bmatrix}&=\begin{bmatrix}
6\\-1
\end{bmatrix}
\end{align*}
\begin{align*}
A &= \begin{bmatrix}
5&4\\ 1&3
\end{bmatrix}\\
D&= \begin{bmatrix}
5&0\\ 0&3
\end{bmatrix}, D^{-1} =\begin{bmatrix}
\frac{1}{5}&0\\ 0&\frac{1}{3}
\end{bmatrix}\\
U &=\begin{bmatrix}
0&4\\ 0&0
\end{bmatrix},
L = \begin{bmatrix}
0&0\\ 1&0
\end{bmatrix}\\
U + L&=\begin{bmatrix}
0&4\\ 1&0
\end{bmatrix}, D + L=\begin{bmatrix}
5&0\\ 1&3
\end{bmatrix}\\
(D+L)^{-1} &= \begin{bmatrix}
\frac{1}{5}&0\\-\frac{1}{15}&\frac{1}{3}
\end{bmatrix}
\end{align*}
\begin{itemize}
\item Jacobi Method \begin{align*}
x^{(1)} &= D^{-1}(b- (U+L)x^{(0)})\\
&=\begin{bmatrix}
\frac{1}{5}&0\\ 0&\frac{1}{3}
\end{bmatrix}\left(\begin{bmatrix}
6\\-1
\end{bmatrix} -\begin{bmatrix}
0&4\\ 1&0
\end{bmatrix}\begin{bmatrix}
0\\0
\end{bmatrix}\right)\\
&=\begin{bmatrix}
\frac{6}{5}\\-\frac{1}{3}
\end{bmatrix}\\
x^{(2)} &= D^{-1}(b- (U+L)x^{(1)})\\
&=\begin{bmatrix}
\frac{1}{5}&0\\ 0&\frac{1}{3}
\end{bmatrix}\left(\begin{bmatrix}
6\\-1
\end{bmatrix} -\begin{bmatrix}
0&4\\ 1&0
\end{bmatrix}\begin{bmatrix}
\frac{6}{5}\\-\frac{1}{3}
\end{bmatrix}\right)\\
&=\begin{bmatrix}
\frac{1}{5}&0\\ 0&\frac{1}{3}
\end{bmatrix}\left(\begin{bmatrix}
6\\-1
\end{bmatrix} -\begin{bmatrix}
-\frac{4}{3}\\\frac{6}{5}
\end{bmatrix}\right)\\
&=\begin{bmatrix}
\frac{1}{5}&0\\ 0&\frac{1}{3}
\end{bmatrix}\begin{bmatrix}
\frac{22}{3}\\-\frac{11}{5}
\end{bmatrix}\\
&=\begin{bmatrix}
\frac{22}{15}\\-\frac{11}{15}
\end{bmatrix}\\
\end{align*}
\item 
Gauss - Seidel Method
\begin{align*}
x^{(1)} &= (D+L)^{-1}(b - Ux^{(0)})\\
&=\begin{bmatrix}
\frac{1}{5} &0 \\ -\frac{1}{15} &\frac{1}{3}
\end{bmatrix}\left(\begin{bmatrix}
6\\-1
\end{bmatrix}-\begin{bmatrix}
0&4\\0&0
\end{bmatrix}\begin{bmatrix}
0\\0
\end{bmatrix}\right)\\
&=\begin{bmatrix}
\frac{6}{5}\\-\frac{11}{15}
\end{bmatrix}\\
x^{(2)} &= (D+L)^{-1}(b - Ux^{(1)})\\
&=\begin{bmatrix}
\frac{1}{5} &0 \\ -\frac{1}{15} &\frac{1}{3}
\end{bmatrix}\left(\begin{bmatrix}
6\\-1
\end{bmatrix}-\begin{bmatrix}
0&4\\0&0
\end{bmatrix}\begin{bmatrix}
\frac{6}{5}\\-\frac{11}{15}
\end{bmatrix}\right)\\
&=\begin{bmatrix}
\frac{1}{5} &0 \\ -\frac{1}{15} &\frac{1}{3}
\end{bmatrix}\left(\begin{bmatrix}
6\\-1
\end{bmatrix}-\begin{bmatrix}
-\frac{44}{15}\\0
\end{bmatrix}\right)\\
&=\begin{bmatrix}
\frac{1}{5} &0 \\ -\frac{1}{15} &\frac{1}{3}
\end{bmatrix}\begin{bmatrix}
\frac{134}{15}\\-1
\end{bmatrix}\\
&=\begin{bmatrix}
\frac{134}{75}\\-\frac{209}{225}
\end{bmatrix}
\end{align*}
\end{itemize}
\item The original problem can be expressed as \begin{align*}
\begin{bmatrix}
1&-8&-2\\1&1&5\\3&-1&1
\end{bmatrix}\begin{bmatrix}
u\\v\\w
\end{bmatrix}&=\begin{bmatrix}
1\\4\\-2
\end{bmatrix}
\end{align*}
And we can change it to diagonally dominant by 
\begin{align*}
\begin{bmatrix}
3&-1&1\\1&-8&-2\\1&1&5
\end{bmatrix}\begin{bmatrix}
u\\v\\w
\end{bmatrix}&=\begin{bmatrix}
-2\\1\\4
\end{bmatrix}
\end{align*}
And we can get \begin{align*}
u&= \frac{-2+v-w}{3}\\
v &=\frac{u-2w-1}{8}\\
w &= \frac{4-u-v}{5}
\end{align*}
\begin{itemize}
\item Jacobi Method
\begin{align*}
x^{(1)}&= \begin{bmatrix}
-\frac{2}{3}\\
-\frac{1}{8}\\
\frac{4}{5}
\end{bmatrix}\\
x^{(2)}&= \begin{bmatrix}
\frac{-2+v^1-w^1}{3}\\
\frac{u^1-2w^1-1}{8}\\
\frac{4-u^1-v^1}{5}
\end{bmatrix}\\
&=\begin{bmatrix}
\dfrac{-2+\frac{-1}{8}-\frac{4}{5}}{3}\\
\dfrac{\frac{-2}{3}-2\cdot \frac{4}{5}-1}{8}\\
\dfrac{4-\frac{-2}{3}-\frac{-1}{8}}{5}
\end{bmatrix}\\
&=\begin{bmatrix}
-\frac{117}{120}\\-\frac{49}{120}\\\frac{23}{24}
\end{bmatrix}
\end{align*}
\item  Gauss - Seidel Method
\begin{align*}
x^{(1)}&= \begin{bmatrix}
\frac{-2+v^0-w^0}{3}\\
\frac{u^1-2w^0-1}{8}\\
\frac{4-u^1-v^1}{5}
\end{bmatrix}\\
&=\begin{bmatrix}
-\frac{2}{3}\\
\frac{-\frac{2}{3}-1}{8}\\
\frac{4+\frac{2}{3}-v^1}{5}
\end{bmatrix}\\
&=\begin{bmatrix}
-\frac{2}{3}\\
-\frac{5}{24}\\
\frac{4+\frac{2}{3}+\frac{5}{24}}{5}
\end{bmatrix}\\
&=\begin{bmatrix}
-\frac{2}{3}\\
-\frac{5}{24}\\
\frac{39}{40}
\end{bmatrix}\\
x^{(2)}&= \begin{bmatrix}
\frac{-2+v^1-w^1}{3}\\
\frac{u^2-2w^1-1}{8}\\
\frac{4-u^2-v^2}{5}
\end{bmatrix}\\
&= \begin{bmatrix}
\frac{-2-\frac{5}{24}-\frac{39}{40}}{3}\\
\frac{u^2-2\frac{39}{40}-1}{8}\\
\frac{4-u^2-v^2}{5}
\end{bmatrix}\\
&= \begin{bmatrix}
-\frac{191}{180}\\ 	
\frac{-\frac{191}{180}-2\frac{39}{40}-1}{8}\\
\frac{4+\frac{191}{180}-v^2}{5}
\end{bmatrix}\\
&= \begin{bmatrix}
-\frac{191}{180}\\
-\frac{361}{720}\\
\frac{4+\frac{191}{180}+\frac{361}{720}}{5}
\end{bmatrix}\\
&= \begin{bmatrix}
-\frac{191}{180}\\
-\frac{361}{720}\\
\frac{89}{80}
\end{bmatrix}\\
\end{align*}
\end{itemize}
\item
The original problem can be expressed as \begin{align*}
\begin{bmatrix}
1&4&0\\
0&1&2\\
4&0&3
\end{bmatrix}\begin{bmatrix}
u\\v\\w
\end{bmatrix}&=\begin{bmatrix}
5\\2\\0
\end{bmatrix}
\end{align*}
And we can change it to diagonally dominant by\begin{align*}
\begin{bmatrix}
4&0&3\\
1&4&0\\
0&1&2
\end{bmatrix}\begin{bmatrix}
u\\v\\w
\end{bmatrix}&=\begin{bmatrix}
0\\5\\2
\end{bmatrix}
\end{align*}
And we can get\begin{align*}
u &= -\frac{3w}{4}\\
v &= \frac{5-u}{4}\\
w&= \frac{2-v}{2}
\end{align*}
\begin{itemize}
\item Jacobi Method
\begin{align*}
x^1 &=\begin{bmatrix}
 -\frac{3w^0}{4}\\
 \frac{5-u^0}{4}\\
 \frac{2-v^0}{2}
\end{bmatrix}\\
&=\begin{bmatrix}
 0\\
 \frac{5}{4}\\
1
\end{bmatrix}\\
x^2 &=\begin{bmatrix}
 -\frac{3w^1}{4}\\
 \frac{5-u^1}{4}\\
 \frac{2-v^1}{2}
\end{bmatrix}\\
&=\begin{bmatrix}
 -\frac{3\cdot 1}{4}\\
 \frac{5}{4}\\
 \frac{2- \frac{5}{4}}{2}
\end{bmatrix}\\
&=\begin{bmatrix}
 -\frac{3}{4}\\
 \frac{5}{4}\\
\frac{3}{8}
\end{bmatrix}\\
\end{align*}
\item Gauss - Seidel Method
\begin{align*}
x^1 &=\begin{bmatrix}
 -\frac{3w^0}{4}\\
 \frac{5-u^1}{4}\\
 \frac{2-v^1}{2}
\end{bmatrix}\\
&=\begin{bmatrix}
 0\\
 \frac{5}{4}\\
 \frac{2-\frac{5}{4}}{2}
\end{bmatrix}\\
&=\begin{bmatrix}
 0\\
 \frac{5}{4}\\
\frac{3}{8}
\end{bmatrix}\\
x^2 &=\begin{bmatrix}
 -\frac{3w^1}{4}\\
 \frac{5-u^2}{4}\\
 \frac{2-v^2}{2}
\end{bmatrix}\\
&=\begin{bmatrix}
 -\frac{3\cdot \frac{3}{8}}{4}\\
 \frac{5-u^2}{4}\\
 \frac{2-v^2}{2}
\end{bmatrix}\\
&=\begin{bmatrix}
 -\frac{9}{32}\\
 \frac{5 +\frac{9}{32}}{4}\\
 \frac{2-v^2}{2}
\end{bmatrix}\\
&=\begin{bmatrix}
 -\frac{9}{32}\\
 \frac{169}{128}\\
 \frac{2- \frac{169}{128}}{2}
\end{bmatrix}\\
&=\begin{bmatrix}
 -\frac{9}{32}\\
 \frac{169}{128}\\
\frac{87}{256}
\end{bmatrix}\\
\end{align*}
\end{itemize}
\end{enumerate}
\item
\begin{enumerate}
\item Jacobi Method:
\begin{itemize}
\item
\begin{align*}
\begin{bmatrix}
3&1\\1&2
\end{bmatrix} x &= b\\
\begin{bmatrix}
3&0\\0&2
\end{bmatrix} x^{k}+ \begin{bmatrix}
0&1\\1&0
\end{bmatrix} x^{k-1}&= b\\
\begin{bmatrix}
3&0\\0&2
\end{bmatrix} x^{k}&= b- \begin{bmatrix}
0&1\\1&0
\end{bmatrix} x^{k-1}\\
\begin{bmatrix}
\frac{1}{3}&0\\0&\frac{1}{2}
\end{bmatrix}\begin{bmatrix}
3&0\\0&2
\end{bmatrix} x^{k}&= \begin{bmatrix}
\frac{1}{3}&0\\0&\frac{1}{2}
\end{bmatrix}\left(b- \begin{bmatrix}
0&1\\1&0
\end{bmatrix} x^{k-1}\right)\\
 x^{k}&= \begin{bmatrix}
\frac{1}{3}&0\\0&\frac{1}{2}
\end{bmatrix}b- \begin{bmatrix}
0&\frac{1}{3}\\\frac{1}{2}&0
\end{bmatrix} x^{k-1}\\
B&=- \begin{bmatrix}
0&\frac{1}{3}\\\frac{1}{2}&0
\end{bmatrix}\\
\end{align*}
\item 
\begin{align*}
 \begin{vmatrix}
-\lambda&-\frac{1}{3}\\-\frac{1}{2}&-\lambda
\end{vmatrix}&=0\\
\lambda^2 &=\frac{1}{6}\\
\rho(B) &=\frac{\sqrt{6}}{6}
\end{align*}
\item
Since $\rho(B) =\frac{\sqrt{6}}{6} < 1$, so converge.
\item
\begin{align*}
\begin{bmatrix}
1&2\\3&1
\end{bmatrix} x &= b\\
\begin{bmatrix}
1&0\\0&1
\end{bmatrix} x^{k}+ \begin{bmatrix}
0&3\\2&0
\end{bmatrix} x^{k-1}&= b\\
\begin{bmatrix}
1&0\\0&1
\end{bmatrix} x^{k}&= b- \begin{bmatrix}
0&3\\2&0
\end{bmatrix} x^{k-1}\\
 x^{k}&= b- \begin{bmatrix}
0&3\\2&0
\end{bmatrix} x^{k-1}\\
B&=- \begin{bmatrix}
0&3\\2&0
\end{bmatrix}\\
\begin{vmatrix}
-\lambda&3\\2&-\lambda
\end{vmatrix}&=0\\
\lambda^2 &= 6\\
\rho(B) &=\sqrt{6}
\end{align*}
Since $\rho(B) =\sqrt{6}> 1$, so diverge.
\end{itemize}
\item Gauss - Seidel Method
\begin{itemize}
\item
\begin{align*}
\begin{bmatrix}
3&1\\1&2
\end{bmatrix} x &= b\\
\begin{bmatrix}
3&0\\1&2
\end{bmatrix} x^{k}+ \begin{bmatrix}
0&1\\0&0
\end{bmatrix} x^{k-1}&= b\\
\begin{bmatrix}
3&0\\1&2
\end{bmatrix} x^{k}&=- \begin{bmatrix}
0&1\\0&0
\end{bmatrix} x^{k-1}+ b\\
\begin{bmatrix}
\frac{1}{3}&0\\-\frac{1}{6}&\frac{1}{2}
\end{bmatrix}\begin{bmatrix}
3&0\\1&2
\end{bmatrix} x^{k}&=- \begin{bmatrix}
\frac{1}{3}&0\\-\frac{1}{6}&\frac{1}{2}
\end{bmatrix}\begin{bmatrix}
0&1\\0&0
\end{bmatrix} x^{k-1}+ \begin{bmatrix}
\frac{1}{3}&0\\-\frac{1}{6}&\frac{1}{2}
\end{bmatrix}b\\
 x^{k}&=\begin{bmatrix}
0&-\frac{1}{3}\\0&\frac{1}{6}
\end{bmatrix} x^{k-1}+ \begin{bmatrix}
\frac{1}{3}&0\\-\frac{1}{6}&\frac{1}{2}
\end{bmatrix}b\\
B &= \begin{bmatrix}
0&-\frac{1}{3}\\0&\frac{1}{6}
\end{bmatrix}
\end{align*}
\item
\begin{align*}
\begin{vmatrix}
-\lambda&-\frac{1}{3}\\0&\frac{1}{6}-\lambda
\end{vmatrix}&=0\\
\lambda &= 0, \frac{1}{6}\\
\rho(B)&= \frac{1}{6}
\end{align*}
\item 
Since $\rho(B)=\frac{1}{6} < 1$, so converge.
\item
\begin{align*}
\begin{bmatrix}
1&2\\3&1
\end{bmatrix} x &= b\\
\begin{bmatrix}
1&0\\3&1
\end{bmatrix} x^{k}+ \begin{bmatrix}
0&2\\0&0
\end{bmatrix} x^{k-1}&= b\\
\begin{bmatrix}
1&0\\3&1
\end{bmatrix} x^{k}&=- \begin{bmatrix}
0&2\\0&0
\end{bmatrix} x^{k-1}+ b\\
\begin{bmatrix}
1&0\\-3&1
\end{bmatrix}\begin{bmatrix}
1&0\\3&1
\end{bmatrix} x^{k}&=- \begin{bmatrix}
1&0\\-3&1
\end{bmatrix}\begin{bmatrix}
0&2\\0&0
\end{bmatrix} x^{k-1}+ \begin{bmatrix}
1&0\\-3&1
\end{bmatrix}b\\
 x^{k}&=\begin{bmatrix}
0&2\\0&-6
\end{bmatrix} x^{k-1}+ \begin{bmatrix}
1&0\\-3&1
\end{bmatrix}b\\
B &= \begin{bmatrix}
0&2\\0&-6
\end{bmatrix}\\
\begin{vmatrix}
-\lambda&2\\0&-6-\lambda
\end{vmatrix}&=0\\
\lambda &= 0, -6\\
\rho(B)&= 6
\end{align*}
Since $\rho(B)=6 >1$, so diverge.
\end{itemize}
\end{enumerate}
\end{enumerate}

\end{document}